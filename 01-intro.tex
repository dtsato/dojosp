\section{Introduction}\label{sec:introduction}

\begin{quote}
In software we do our practicing on the job, and that’s why we make mistakes on the job.
We need to find ways of splitting the practice from the profession. We need practice sessions.

--Dave Thomas
\end{quote}

The idea of a \emph{Code Kata} was initially proposed by Dave Thomas as an exercise where programmers
could write throwaway code to practice their craft outside of a working environment~\cite{DaveThomas}.
Laurent Bossavit later proposed the idea of a \emph{Coding Dojo}: a session where a group of programmers
would gather to solve the \emph{Code Kata} togheter~\cite{Bossavit}. Although the session is organized
around a programming challenge, the main goal of a \emph{Coding Dojo} is to learn from others and
improve design and coding skills through deliberate practice. This creates a learning environment where
Agile technical practices, such as those proposed by Extreme Programming (XP)~\cite{XP2E}, can be shared.

This report describes the authors' experience of founding and running a \emph{Coding Dojo} in
São Paulo, Brazil. Section~\ref{sec:dojo} will present the concept and rules of a \emph{Coding Dojo} and
the tailored process to conduct the sessions in São Paulo, improved over time by retrospectives.
Section~\ref{sec:lessons_learned} will present lessons learned from the weekly meetings being held since
the first session in July, 2007. Section~\ref{sec:learning} will discuss the aspects of a \emph{Coding Dojo}
that foster learning and tacit knowledge sharing, concluding in Section~\ref{sec:conclusion}.