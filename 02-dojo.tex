\section{Coding Dojo}\label{sec:dojo}

A \emph{Coding Dojo} is a periodic meeting (usually weekly) organized around a programming challenge
where people are encouraged to participate and share their coding skills with the audience while
solving the problem. The meeting is held in a room with enough space for all the participants and
usually requires only a projector and a computer or laptop. Having whiteboard space for sketching
and design discussions is also valuable.

The main principles of the \emph{Coding Dojo} are to create a \textbf{Safe Environment} which is
collaborative, inclusive, and non-competitive where people can be \textbf{Continuously Learning}.
Some of the XP principles align nicely with that~\cite{XP2E}, such as \textbf{Failure} -- it is OK
to fail when learning something new -- \textbf{Redundancy} -- one can always gain new insights
when tackling the same problem with different strategies -- and \textbf{Baby Steps} -- each step
towards the solution should be small enough so that everybody can comprehend and replicate it later.

Rules

Practices

In an inclusive and collaborative environment, the participants discuss and
practice a wide range of topics, such as: TDD/BDD, Agile, refactoring, pair programming, OO, design, Algorithms, different programming languages, paradigms, and frameworks.

\cite{DojoWiki}

\subsection{Coding Dojo@SP: Numbers and Processes}\label{subsec:dojosp}

session agenda, \# of participants, \# of meetings, Languages utilized, ...

