\section{Coding Dojo}\label{sec:dojo}

A \emph{Coding Dojo} is a periodic meeting (usually weekly) organized around a programming challenge
where people are encouraged to participate and share their coding skills with the audience while
solving the problem. The main principles of the \emph{Coding Dojo} are to create a
\textbf{Safe Environment} which is collaborative, inclusive, and non-competitive where people can
be \textbf{Continuously Learning}. Some of the XP principles align nicely with that~\cite{XP2E}, such
as \textbf{Failure} -- it is OK to fail when learning something new -- \textbf{Redundancy} -- one can
always gain new insights when tackling the same problem with different strategies -- and
\textbf{Baby Steps} -- each step towards the solution should be small enough so that everybody can
comprehend and replicate it later.

There are some general rules that allow the \emph{Coding Dojo} session to be productive and to flow.
The meeting is held in a room with enough space for all the participants and usually requires only a
projector and a computer or laptop. Having whiteboard space for sketching and design discussions is also
valuable. The participants are encouraged to develop the solution using Test-Driven Development
(TDD)~\cite{TDD} and are free to choose whichever programming language they prefer. There are two main
meeting formats:

\begin{itemize}
	\item \textbf{Prepared Kata}: In this format, someone has already solved the proposed \emph{Kata}
	prior to the meeting (alone or in group) and the solution is presented to the audience during the session.
	Instead of showing the final code and tests, the presenters start from scratch, explaining each step
	and allowing the other participants to ask questions or make suggestions. The session goal is that
	everyone should be able to reproduce the steps and solve the same problem after the meeting.
	
	\item \textbf{Randori}: In this format, the participants solve the problem together, following TDD and
	Pair Programming in time-boxed rounds (usually between 5 and 7 minutes). At the end of each turn, the
	pilot joins the audience, the co-pilot becomes pilot, and a new co-pilot joins the pair from the
	audience. An extra rule is that discussions and suggestions should be only given when the pair arrives
	in a green bar, with all the current tests passing. The reason is that, while on a red bar, the pair should
	focus and work together to get the tests passing. The audience can always suggest refactorings and
	optimizations on a green bar.
\end{itemize}

These formats allow the creation of an environment where participants can discuss and
practice a wide range of topics, such as: TDD, Behaviour-Driven Development, Agile, Refactoring, Pair Programming,
Object-Oriented Design, Algorithms, different programming languages, paradigms, and frameworks.

\subsection{Coding Dojo@SP: Numbers and Processes}\label{subsec:dojosp}

session agenda, \# of participants, \# of meetings, Languages utilized, ...\cite{DojoWiki}


