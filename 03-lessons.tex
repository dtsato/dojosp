\section{Lessons Learned}\label{sec:lessons_learned}

After over 6 months of weekly meetings, the authors have identified
aspects of the Dojo that went well, aspects that went less well and
aspects that still are puzzles to understand.

Since the sessions are being held, the authors could identify what
practices and rules went well (Subsection \ref{ssub:well}) but also
found out that some things work less well (\ref{ssub:less_well}) than
expected. Finally, applying the practice to different audiences and in
different contexts, the authors discovered some unaddressed issues
(\ref{ssub:puzzles}).

\subsubsection{What went well?}\label{ssub:well}

\noindent
\textbf{The goal is not to finish}

When the Dojo started, the participants agreed that one of the goals
was to learn different algorithms and approaches to problem solving.

On the second meeting, we attempted to solve a problem, and the
time-boxed rounds became a race of who could produce more code and get
closer to solving the problem. The coding happened really fast, and
soon some participants could not keep up with what changes were made
and why. In the aftermath of this meeting, the group decided that
finishing the problem should never be a goal of the meeting. More than
that, it was agreed that not writing the entire solution was OK, as
long as we could learn something from the coding session. ``It's OK
not to finish'' has been one of our principles ever since, and we
repeat it whenever someone forgets or doesn't know.

With that principle clear, the Dojo participants take their time
writing code and undestanding it, and the group often does not finish
coding all aspects of the problems proposed.

\noindent
\textbf{Retrospectives and action items}

The retrospective at the end of each meeting allows Dojo participants
to evolve and the Dojo itself to improve.

As described in the previous section, every Coding Dojo meeting is
ended with a short retrospective. The participants receive red and
yellow sticky cards and write positive and negative aspects of the
meeting that just ended. In the begining we followed the standard
retrospective format, asking ``What do we want to keep?''  for
positive aspects and ``What can we improve?'' for negative
aspects. After some time, we found that the items written by people
focused on environmental aspects of the Dojo such as the room, the
people present, etc. In hopes of motivating more reflexion and
instrospection, a different pair of questions was sugested: ``What
have we learned?'' for positive aspects and ``What has hindered
learning?'' for negative aspects.

The positive aspects question motivates people to think about what
they have actually learned during this session. Since the primary goal
of the Dojo is learning and practicing, this is an effective way to...

{\bf** Basicamente, quero falar que o ``pensar sobre o que aprendemos''
ajuda a gente a aprender de fato. Se nao pensarmos, aprendemos menos.}

The negative aspects are also discussed from a practical point of
view. After reviewing and identifying the main negative aspects, the
group discussed possible solutions or actions to prevent the ``bad
things'' from happening again.

\noindent
\textbf{Time-boxing}

The Sao Paulo Coding Dojo has always used 7 minutes time-boxes for
Randori sessions. Since the beginning, the participantes have felt
that 5 minutes was too short to be good. However, for a long time the
group also disrespected a bit the time-boxes. That is, if a pair was
in the middle of writing a piece of test or a refactoring, and the
group considered this activity to be short, the pair was allowed to
finish the current code before switching. At first this took 1 or 2
minutes more, but this ``overtime'' gradually increased until there
was no more time box, but a minimum time for pairing.

This actually made it difficult for everyone to be focused on the
screen - the longer a pair stayed at the front, less and less people
payed attention to them. As a result, the group decided to adopt
\textbf{really scrict} time-boxes. When the timer rings, no matter
what else, the pair is switched. This has made meetings more dynamic
and easy to follow, although some focus problems remain, as will be
discussed in the next subsection \ref{ssub:less_well}.

[One side-effect of this approach is that if some discussion happens
between the group, the current pair has less coding time. The
participants have not yet found a solution to this, but some ideas
have been suggested and should be attempted at future meetings.]

\noindent
\textbf{Information radiators}

\noindent
\textbf{Communication}

\noindent
\textbf{Inspiration for the meeting}

\subsubsection{What went less well?}\label{ssub:less_well}

\noindent
\textbf{Moderating brazilians (hard not to speak on red)}

One of the rules imported from international Dojos is that, during a
Randori, the audience should not speak when the tests are red. Red
time is when the current pair is supposed to practice and make the
tests pass, and only if they ask for help should the other
participants give suggestions.

However, from the begining this has been a hard practice to follow at
de Sao Paulo Coding Dojo.

One of the problems the Dojo participants have faced from the begining
is that people talk at bad moments. The authors believe this is
related to cultural aspects of the group. Brazilian people are very
communicative, and 

\noindent
\textbf{TDD/BDD and algorithms}

\noindent
\textbf{Balancing randoris and prepared katas}

\noindent
\textbf{Programming environment}

\subsubsection{What is still puzzling?}\label{ssub:puzzles}

\noindent
\textbf{How to reach a wider audience?}

\noindent
\textbf{How to share our efforts with the community?}

\noindent
\textbf{How to keep attendees engaged?}
