\section{Conclusion}\label{sec:conclusion}

This report shares the experiences of running a \emph{Coding Dojo} at the University of São Paulo and, more
recently, at ThoughtWorks. The process and roles used to conduct the meetings were improved through
retrospectives based on the participants' feedback. By sharing the lessons learned from this experience, the
authors expect that this learning tool can be applied to different contexts, encouraging more people to
start their own \emph{Coding Dojos}. Finally, the role of a \emph{Coding Dojo} in the learning process
was discussed, showing how students at different skill levels can use deliberate practice to improve and to
share knowledge with a wider group.

\section*{Acknowledgements}

The authors would like to thank Prof. Dr. Alfredo Goldman and Prof. Dr. Fabio Kon for supporting the São Paulo
\emph{Coding Dojo} sessions and for providing the environment and resources needed to run the sessions at the
University of São Paulo. A special thanks goes also to Fabricio de Sousa for helping the organization of several
sessions and for presenting the \emph{Coding Dojo} concept in different conferences. The authors are also very
grateful for learning and sharing experiences with all the \emph{Coding Dojo} participants.  