\documentclass[times, 10pt,twocolumn]{article} 
\usepackage{latex8}
\usepackage[utf8]{inputenc}
\usepackage{verbatim}
\usepackage{times}
\usepackage{url}
\usepackage{graphicx}

%------------------------------------------------------------------------- 
% take the % away on next line to produce the final camera-ready version 
\pagestyle{empty}

%------------------------------------------------------------------------- 
\begin{document}

\title{Coding Dojo: an environment for learning and sharing Agile practices}

\author{Danilo Sato\\
\textit{ThoughtWorks Limited}\\ 
{\tt dsato@thoughtworks.com}\\
\and
Hugo Corbucci, Mariana Bravo\\
\textit{Department of Computer Science}\\
\textit{University of São Paulo, Brazil}\\
{\tt \{corbucci, marivb\}@ime.usp.br}
}

\maketitle
\thispagestyle{empty}

\begin{abstract}
  A Coding Dojo is a meeting where a group of programmers gets together to learn, practice,
and share experiences. This report describes the authors' experience of creating and running an
active Coding Dojo in São Paulo, Brazil, sharing the lessons learned from the experience.
The role of the Dojo in the learning process is discussed, showing how it creates an environment
for fostering and sharing Agile practices such as Test-Driven Development, Refactoring and Pair
Programming, among others.
\end{abstract}

\section{Introduction}\label{sec:introduction}

\begin{quote}
In software we do our practicing on the job, and that’s why we make mistakes on the job.
We need to find ways of splitting the practice from the profession. We need practice sessions.

--Dave Thomas
\end{quote}

The idea of a \emph{Code Kata} was initially proposed by Dave Thomas as an exercise where programmers
could write throwaway code to practice their craft outside of a working environment~\cite{DaveThomas}.
Laurent Bossavit later proposed the idea of a \emph{Coding Dojo}: a session where a group of programmers
would gather to solve the \emph{Code Kata} togheter~\cite{Bossavit}. Although the session is organized
around a programming challenge, the main goal of a \emph{Coding Dojo} is to learn from others and
improve design and coding skills through deliberate practice. This creates a learning environment where
technical practices such as those proposed by Extreme Programming (XP)~\cite{XP2E} can be shared.

This report describes the authors' experience of founding and running a \emph{Coding Dojo} in
São Paulo, Brazil. Section~\ref{sec:dojo} will present the tailored process to conduct the sessions,
improved over time by retrospectives. Section~\ref{sec:lessons_learned} will present lessons
learned from the weekly meetings being held since the first session in July, 2007.
Section~\ref{sec:learning} will discuss the aspects of a \emph{Coding Dojo} that foster learning and
tacit knowledge sharing, concluding in Section~\ref{sec:conclusion}.
\section{Coding Dojo}\label{sec:dojo}

A \emph{Coding Dojo} is a periodic meeting (usually weekly) organized around a programming challenge
where people are encouraged to participate and share their coding skills with the audience while
solving the problem. The main principles of the \emph{Coding Dojo} are to create a
\textbf{Safe Environment} which is collaborative, inclusive, and non-competitive where people can
be \textbf{Continuously Learning}. Some of the XP principles align nicely with that~\cite{XP2E}, such
as \textbf{Failure} -- it is OK to fail when learning something new -- \textbf{Redundancy} -- one can
always gain new insights when tackling the same problem with different strategies -- and
\textbf{Baby Steps} -- each step towards the solution should be small enough so that everybody can
comprehend and replicate it later.

There are some general rules that allow the \emph{Coding Dojo} session to be productive and to flow.
The meeting is held in a room with enough space for all the participants and usually requires only a
projector and a computer or laptop. Having whiteboard space for sketching and design discussions is also
valuable. The participants are encouraged to develop the solution using Test-Driven Development
(TDD)~\cite{TDD} and are free to choose whichever programming language they prefer. There are two main
meeting formats:

\begin{itemize}
	\item \textbf{Prepared Kata}: In this format, someone has already solved the proposed \emph{Kata}
	prior to the meeting (alone or in group) and the solution is presented to the audience during the session.
	Instead of showing the final code and tests, the presenters start from scratch, explaining each step
	and allowing the other participants to ask questions or make suggestions. The session goal is that
	everyone should be able to reproduce the steps and solve the same problem after the meeting.
	
	\item \textbf{Randori}: In this format, the participants solve the problem together, following TDD and
	Pair Programming in time-boxed rounds (usually between 5 and 7 minutes). At the end of each turn, the
	pilot joins the audience, the co-pilot becomes pilot, and a new co-pilot joins the pair from the
	audience. An extra rule is that discussions and suggestions should only be given when the pair arrives
	in a green bar, with all the current tests passing. The reason is that, while on a red bar, the pair should
	focus and work together to pass the tests. The audience can always suggest refactorings and
	optimizations on a green bar.
\end{itemize}

These formats allow the creation of an environment where participants can discuss and
practice a wide range of topics, such as: TDD, Behaviour-Driven Development, Agile, Refactoring, Pair Programming,
Object-Oriented Design, Algorithms, different programming languages, paradigms, and frameworks.

\subsection{Coding Dojo@SP: History and Process}\label{subsec:dojosp}

The meetings of the São Paulo \emph{Coding Dojo} started in $12^{th}$ of July, 2007 and have been held weekly
since then in the Institute of Mathematics and Statistics of the University of São Paulo. Some extra sessions were
done during the University holidays and most of the session reports are available on the international
\emph{Coding Dojo} wiki\cite{DojoWiki}. The number of participants varied from 3 to 16 and their skill level ranged
from undergraduate students to experienced programmers.

On the first meeting, the participants were asked to fill index cards with their expectations and personal
interests in attending the sessions. An affinity map was built with that information and the three main interests
were to practice problem solving skills, to learn different ways and algorithms to solve the challenges, and to
learn new programming languages. Some of the sessions were highly focused on design problems and algorithms, which
left less time for writing code, but the participants liked to learn from the discussions. On the other hand, the
majority of the sessions required less design and algorithms discussion, leaving more time to write code and
allowing the participants to experiment with a wide range of programming languages, such as Java, C, Ruby, Python,
Lua, and Smalltalk.

The sessions usually follow the same process:

\begin{itemize}
	\item \textbf{Problem Choosing}: Before the meeting, the participants receive an e-mail with 3 to 5 options
	of problems to be solved. The problems are chosen from several sources (such as Ruby 
	Quiz\footnote{\url{http://www.rubyquiz.com/}}, Programming
	Challenges\footnote{\url{http://www.programming-challenges.com/}}, UVa\footnote{\url{http://acm.uva.es/p/}},
	and SPOJ\footnote{\url{http://www.spoj.pl/}}). Each option is briefly presented and the participants vote
	on which problem will be solved in the meeting. This usually takes 5 to 10 minutes.
	
	\item \textbf{Problem Discussion}: Once the problem is chosen, the group discusses the different approaches
	to solving it, usually ending up with an agreed approach and a list of TO-DO items, as proposed by Kent
	Beck~\cite{TDD}, to guide the pairs during the implementation. This usually takes 10 to 20 minutes, but there
	were meetings when the group spent the entire meeting discussing algorithms and possible approaches to a
	complex problem.
	
	\item \textbf{Coding Session}: With an agreed approach to solve the problem, the participants start the
	coding session in one of the two formats -- a \emph{Prepared Kata} or a \emph{Randori}. They should practice
	Pair Programming and Test-Driven Development as a general rule. This usually takes 1 to 2 hours.
	
	\item \textbf{Retrospective}: At the final 10 to 20 minutes of the session, the participants stop coding
	(even if the problem was not completely solved) to reflect on the experience and share the learned lessons
	with the group. This is also a good time to discuss what could be improved and to come up with action items
	for the next meeting.
\end{itemize}

Finally, the São Paulo \emph{Coding Dojo} came up with two special roles that can be rotated between participants,
but that are very important to organize and to make sure the meetings continue to happen. The \textbf{Moderator}
or \textbf{Organizer} is responsible for what happens before, during, and after the meeting. He handles tasks
such as reserving the meeting room, sending reminders and options of problems to be solved, setting up the
computer and projector prior to the meeting, moderating discussions, conducting the retrospective, and cleaning
up the room after the session. The \textbf{Scribe} is responsible for publishing the results of the session and
sharing it with the people that could not attend the meeting. He handles tasks such as posting the session report
to the wiki, publishing the final source code to the group, sometimes taking photos, and documenting the results
of the retrospective.

\section{Lessons Learned}\label{sec:lessons_learned}

After over 6 months of weekly meetings, the authors have identified
aspects of the Dojo that went well, aspects that went less well and
aspects that still are puzzles to understand.

Since the sessions are being held, the authors could identify what
practices and rules went well (Subsection \ref{ssub:well}) but also
found out that some things work less well (\ref{ssub:less_well}) than
expected. Finally, applying the practice to different audiences and in
different contexts, the authors discovered some unaddressed issues
(\ref{ssub:puzzles}).

\subsubsection{What went well?}\label{ssub:well}

\noindent
\textbf{The goal is not to finish}

When the Dojo started, the participants agreed that one of the goals
was to learn different algorithms and approaches to problem solving.

On the second meeting, we attempted to solve a problem, and the
time-boxed rounds became a race of who could produce more code and get
closer to solving the problem. The coding happened really fast, and
soon some participants could not keep up with what changes were made
and why. In the aftermath of this meeting, the group decided that
finishing the problem should never be a goal of the meeting. More than
that, it was agreed that not writing the entire solution was OK, as
long as we could learn something from the coding session. ``It's OK
not to finish'' has been one of our principles ever since, and we
repeat it whenever someone forgets or doesn't know.

With that principle clear, the Dojo participants take their time
writing code and undestanding it, and the group often does not finish
coding all aspects of the problems proposed.

\noindent
\textbf{Retrospectives and action items}

The retrospective at the end of each meeting allows Dojo participants
to evolve and the Dojo itself to improve.

As described in the previous section, every Coding Dojo meeting is
ended with a short retrospective. The participants receive red and
yellow sticky cards and write positive and negative aspects of the
meeting that just ended. In the begining we followed the standard
retrospective format, asking ``What do we want to keep?''  for
positive aspects and ``What can we improve?'' for negative
aspects. After some time, we found that the items written by people
focused on environmental aspects of the Dojo such as the room, the
people present, etc. In hopes of motivating more reflexion and
instrospection, a different pair of questions was sugested: ``What
have we learned?'' for positive aspects and ``What has hindered
learning?'' for negative aspects.

The positive aspects question motivates people to think about what
they have actually learned during this session. Since the primary goal
of the Dojo is learning and practicing, this is an effective way to...

{\bf** Basicamente, quero falar que o ``pensar sobre o que aprendemos''
ajuda a gente a aprender de fato. Se nao pensarmos, aprendemos menos.}

The negative aspects are also discussed from a practical point of
view. After reviewing and identifying the main negative aspects, the
group discussed possible solutions or actions to prevent the ``bad
things'' from happening again.

\noindent
\textbf{Time-boxing}

The Sao Paulo Coding Dojo has always used 7 minutes time-boxes for
Randori sessions. Since the beginning, the participantes have felt
that 5 minutes was too short to be good. However, for a long time the
group also disrespected a bit the time-boxes. That is, if a pair was
in the middle of writing a piece of test or a refactoring, and the
group considered this activity to be short, the pair was allowed to
finish the current code before switching. At first this took 1 or 2
minutes more, but this ``overtime'' gradually increased until there
was no more time box, but a minimum time for pairing.

This actually made it difficult for everyone to be focused on the
screen - the longer a pair stayed at the front, less and less people
payed attention to them. As a result, the group decided to adopt
\textbf{really scrict} time-boxes. When the timer rings, no matter
what else, the pair is switched. This has made meetings more dynamic
and easy to follow, although some focus problems remain, as will be
discussed in the next subsection \ref{ssub:less_well}.

[One side-effect of this approach is that if some discussion happens
between the group, the current pair has less coding time. The
participants have not yet found a solution to this, but some ideas
have been suggested and should be attempted at future meetings.]

\noindent
\textbf{Information radiators}

\noindent
\textbf{Communication}

\noindent
\textbf{Inspiration for the meeting}

\subsubsection{What went less well?}\label{ssub:less_well}

\noindent
\textbf{Moderating brazilians (hard not to speak on red)}

One of the rules imported from international Dojos is that, during a
Randori, the audience should not speak when the tests are red. Red
time is when the current pair is supposed to practice and make the
tests pass, and only if they ask for help should the other
participants give suggestions.

However, from the begining this has been a hard practice to follow at
de Sao Paulo Coding Dojo.

One of the problems the Dojo participants have faced from the begining
is that people talk at bad moments. The authors believe this is
related to cultural aspects of the group. Brazilian people are very
communicative, and 

\noindent
\textbf{TDD/BDD and algorithms}

\noindent
\textbf{Balancing randoris and prepared katas}

\noindent
\textbf{Programming environment}

\subsubsection{What is still puzzling?}\label{ssub:puzzles}

\noindent
\textbf{How to reach a wider audience?}

\noindent
\textbf{How to share our efforts with the community?}

\noindent
\textbf{How to keep attendees engaged?}

\section{Dojo and Learning}\label{sec:learning}

The main goal of a \emph{Coding Dojo} is learning through practice. Like a pianist
plays scales and a martial arts student practices basic moves, the \emph{Code Katas}
serve as focused exercises that allow the participants to improve on specific skills.
Ericsson et al. studied what influences the acquisition of expertise in
different domains such as music, chess, and sports~\cite{DeliberatePractice}. They found
that deliberate practice over a long period of time (usually more than 10 years) is at the
heart of attaining expertise. Their empirical study shows that experts carefully schedule
deliberate practice and limit its duration to avoid exhaustion and burnout. Although it
takes time to become an expert, the role of deliberate practice is still important through
the learning process.

The Dreyfus Model of skill acquisition defines five developmental stages when learning
a new skill: novice, competence, proficiency, expertise, and mastery~\cite{Dreyfus}. A
\emph{novice} needs a set of pre-defined rules that he can apply to situations without
previous experience on the domain. \emph{Competence} comes with experience, when the student
can identify recurring patterns and understand his environment. With increased practice and
experience, a \emph{proficient} student starts to question the guidelines and is able to
apply different rules considering longer term consequences. Once the repertoire of
experienced situations becomes so vast, an \emph{expert} student is able to intuitively
trigger the appropriate action for a specific situation. According to the Dreyfus model,
there is no higher level of mental capacity than expertise, but there are moments when an
expert can cease to pay conscious attention to his performance and still produce the
appropriate perspective and its associated action, reaching a stage of \emph{mastery}.

Although the \emph{Coding Dojo} can not provide the intuition and unconscious competence
required to achieve expertise and mastery, the deliberate practice of Agile practices
and coding skills can help participants to go from novice to proficient. Also, since
there is no single master for every subject, participants of different levels can
share their knowledge and improve the learning experience of the whole group.

\subsection{Dojo at the University}

Running the \emph{Coding Dojo} at the University gave the authors a pleasant example of
the good impacts of the sessions in one particular student. One of the attendees joined the São
Paulo \emph{Coding Dojo} since the first sessions,
when he had just finished his first semester in Computer Science. He is now on his third semester
and most of his assignments are done using TDD no matter what language is being used. His latest
work involved implementing sparse matrices with common operations in C. He decided by himself to
implement it using TDD and a simple testing library developed during a \emph{Coding Dojo}
session~\cite{Dojo31}. He was able to write clear code with full test coverage. His ability to
identify and pin down the required tests to drive the correct implementation far surpasses his
colleagues'.

He has been showing strong evidences that the knowledge and practices obtained from the
\emph{Coding Dojo} can be absorbed and understood regardless of prior experience on the
subject. Since such testing practices are not part of the regular class' program, it shows
how the participation on the \emph{Coding Dojo} can help a novice to become competent. Practices
that were just followed as rules in the initial sessions became more natural and could be applied
to different contexts and situations. It also shows that the informal, non-directed and non-rigid
learning experience can be effective and complement more traditional teaching methods.

\subsection{Dojo at ThoughtWorks}

More recently, one of the authors had the experience of running a \emph{Coding Dojo}
in a different environment: inside a company. He took over the responsibility of running a
bi-weekly meeting called ``Ruby Tuesdays''. The session's goal was to share knowledge between
expert and novice developers in regards to the Ruby programming language. Although the focus was on
a specific programming language, when the author became the moderator, he made a presentation
and suggested the use of a \emph{Coding Dojo} format for the meetings.

So far the results are very positive. The use of a more structured format allowed the session to
flow better and the use of a single projector proved to help everyone follow the same train of
thought. The retrospective at the end is also helpful to consolidate the lessons learned and to
discuss what can be improved for the next meeting. Running an internal \emph{Coding Dojo} within
a company can help developers to share their interests in particular concepts and practices, allowing
the rest of the organization to experience the benefits of applying different techniques. It also
creates a safe environment, free of normal project pressure, allowing them to conduct controlled
experiments before applying the practices on their day-to-day work.
\section{Conclusion}\label{sec:conclusion}

This report shares the experiences of running a \emph{Coding Dojo} at the University of São Paulo and, more
recently, at ThoughtWorks. The process and roles used to conduct the meetings were improved through
retrospectives based on the participants' feedback. By sharing the lessons learned from this experience, the
authors expect that this learning tool can be applied to different contexts, encouraging more people to
start their own \emph{Coding Dojos}. Finally, the role of a \emph{Coding Dojo} in the learning process
was discussed, showing how students at different skill levels can use deliberate practice to improve and to
share knowledge with a wider group.

\bibliographystyle{latex8}
\bibliography{dojosp}

\end{document}